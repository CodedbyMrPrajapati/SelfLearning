\documentclass[12pt]{article}
\usepackage{amsmath}
\usepackage{graphicx}
\usepackage{enumitem}
\usepackage{geometry}
\geometry{margin=1in}

\title{Monte Carlo Control Based Reinforcement Learning for Snake}
\author{}
\date{}

\begin{document}
\maketitle

\section{Introduction}
This project implements a Reinforcement Learning (RL) agent for the Snake game using
\textbf{Monte Carlo Control with $\epsilon$-soft policies}.
The objective is to learn a policy that maximizes the snake length
by eating food while avoiding collisions.
Episodes correspond to full games until death or stagnation.

\section{Environment and State Representation}
The game uses a grid of size $32\times 16$ (for a $1024\times 512$ pixel window with 32px tiles).
At each timestep, the snake can choose one of three actions:
\[
A = \{-1 \;(\text{left}),\; 0 \;(\text{forward}),\; 1 \;(\text{right})\}.
\]

\subsection{Final State Representation}
The agent observes a compact 6-dimensional binary feature vector:

\begin{align*}
s = (&d_L, d_F, d_R, f_L, f_F, f_R)
\end{align*}

Where:

\begin{itemize}[itemsep=0pt]
    \item $d_L, d_F, d_R$: danger left, front, right (1 if collision will occur)
    \item $f_L, f_F, f_R$: food direction relative to snake's heading, computed using cross/dot products
\end{itemize}

This results in a discrete state space of at most $2^6 = 64$ states,
making Monte Carlo feasible.

\section{Reward Structure}

\begin{itemize}[itemsep=3pt]
    \item \textbf{Food Reward:} $+10$
    \item \textbf{Death Penalty:} $-20$
    \item \textbf{Food Distance Reward (shaping):}
    \[
        r_\text{shape} = 0.2 \cdot \text{sign}(d_\text{old} - d_\text{new})
    \]
    \item \textbf{Stagnation Penalty:} $-0.1$ per step
\end{itemize}

This shaping accelerates learning without changing optimality.

\section{Monte Carlo Control Algorithm}

We use \textbf{first-visit, every-episode MC control} with incremental updates:
\[
Q(s,a) \leftarrow Q(s,a) + \alpha \left(G - Q(s,a)\right)
\]
with $\alpha = 0.1$.

The policy is an $\epsilon$-soft greedy policy:
\[
\pi(a|s) =
\begin{cases}
1-\epsilon + \frac{\epsilon}{3}, & a = \arg\max Q(s,a) \\
\frac{\epsilon}{3}, & \text{otherwise}
\end{cases}
\]

\section{Exploration Schedule}
A decaying $\epsilon$ value is used:
\[
\epsilon_{k+1} = \epsilon_k \cdot 0.9999
\]
Starting from $\epsilon = 0.5$, gradually stabilizing into exploitation.

\section{Stagnation Handling}
To prevent infinite loops, a maximum step length is used:
\[
\text{MAX\_STEP} = 20 + 10 \cdot \text{score}
\]

This allows longer exploration as the snake grows.

\section{Training Observations}

\subsection{Score Progression}
The score plot shows a slow but steady increase from $\approx 1$ to $\approx 12$ over 100{,}000 episodes.

\subsection{Return ($G$) Growth}
Returns improve as the snake learns to:
\begin{itemize}
    \item avoid walls more effectively
    \item reduce unnecessary loops
    \item approach food reliably
\end{itemize}

\subsection{Action Distribution}
Initially, actions are random.
Gradually, the agent prefers:
\begin{itemize}
    \item turning right (dominant)
    \item forward moves increase as it learns corridors
\end{itemize}

Left turns remain sparse—indicating slight exploitation bias.

\subsection{Q-value Evolution}
The $Q_{\max}$ curve approaches values $\approx 60$,
while $Q_{\min}$ stabilizes at $\approx -20$.
Mean Q-value slowly increases toward $0$,
indicating more balanced evaluations.

\subsection{Distance Trends}
The ratio of ``moving closer" to ``moving farther'' improves significantly,
confirming food-shaping effectiveness.

\section{Known Limitations}
\begin{itemize}
\item Monte Carlo learning is slow because it updates only at episode end.
\item Policy may converge to a ``loopy but safe'' strategy.
\item No long-term planning (e.g., tail-chasing for space).
\item Not optimal for high scores without neural function approximation.
\end{itemize}

\section{Conclusion}
Despite using a very small state space and no temporal bootstrapping,
Monte Carlo control successfully learns:
\begin{itemize}
    \item stable survival behavior,
    \item moderate food-seeking,
    \item consistent returns,
    \item and structured action usage.
\end{itemize}

For future improvements:
\begin{itemize}
    \item switch to n-step TD $(n=5)$,
    \item introduce reward for ``alive with large space'',
    \item use full-grid neural encodings.
\end{itemize}

\end{document}
